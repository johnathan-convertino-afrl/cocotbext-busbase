\begin{titlepage}
  \begin{center}

  {\Huge cocotbext busbase}

  \vspace{25mm}

  \includegraphics[width=0.90\textwidth,height=\textheight,keepaspectratio]{img/AFRL.png}

  \vspace{25mm}

  \today

  \vspace{15mm}

  {\Large Jay Convertino}

  \end{center}
\end{titlepage}

\tableofcontents

\newpage

\section{Usage}

\subsection{Introduction}

\par
Cocotb extension to provide base of bus methods.

\subsection{Dependencies}

\par
The following are the dependencies of the cores.

\begin{itemize}
  \item iverilog (simulation)
  \item cocotb (simulation)
  \item cocotb-bus (simulation)
\end{itemize}

\subsection{In a Simulation}
\par
Below is a simple example of creating a basic master for a bus using busmaster (see test.py for usable example).
\begin{lstlisting}[language=Python]
# Class: basictrans
# create an object that associates a data member and address for operation.
class basictrans(transaction):
    def __init__(self, address, data=None):
        self.address = address
        self.data = data

# Class: basicMaster
# basic bus master
class basicMaster(busbase):
  # Variable: _signals
  # List of signals that are required
  _signals = ["addr", "we", "cs", "data"]

  # Constructor: __init__
  # Setup defaults and call base class constructor.
  def __init__(self, entity, name, clock, reset, *args, **kwargs):
    super().__init__(entity, name, clock, *args, **kwargs)

    self.log.info("BASIC Master")
    self.log.info("Copyright (c) 2025 Jay Convertino")
    self.log.info("https://github.com/johnathan-convertino-afrl/cocotbext-busbase")

    self._reset = reset

    self.bus.addr.setimmediatevalue(0)
    self.bus.data.setimmediatevalue(0)

  # Function: read
  # Read from a address and return data
  async def read(self, address):
    trans = None
    if(isinstance(address, list)):
      temp = []
      for a in address:
        temp.append(basictrans(a))
      temp = await self.read_trans(temp)
      #need a return with the data list only. This is only a guess at this point
      return [temp[i].data for i in range(len(temp))]
    else:
      trans = await self.read_trans(basictrans(address))
      return trans.data

  # Function: write
  # Write to a address some data
  async def write(self, address, data):
    if(isinstance(address, list) or isinstance(data, list)):
      if(len(address) != len(data)):
        self.log.error(f'Address and data vector must be the same length')
      temp = []
      for i in range(len(address)):
        temp.append(basictrans(address[i], data[i]))
      await self.write_trans(temp)
    else:
      await self.write_trans(basictrans(address, data))

  # Function: _check_type
  # Check and make sure we are only sending 2 bytes at a time and that it is a bytes/bytearray
  def _check_type(self, trans):
      if(not isinstance(trans, basictrans)):
          self.log.error(f'Transaction must be of type: {type(basictrans)}')
          return False

      return True

  # Method: _run
  # _run thread that deals with read and write.
  async def _run(self):
    self.active = False

    trans = None

    while True:

      if self._reset.value:
        self.bus.we.setimmediatevalue(0)
        self.bus.cs.setimmediatevalue(0)
        await RisingEdge(self.clock)
        continue

      if not self.wqueue.empty():
        self.active = True
        while self.active:
          trans = await self.wqueue.get()
          self.bus.we.setimmediatevalue(1)
          self.bus.cs.setimmediatevalue(1)
          self.bus.addr.setimmediatevalue(trans.address)
          self.bus.data.setimmediatevalue(trans.data)
          self._idle_write.set()
          await RisingEdge(self.clock)

          self.active = not self.wqueue.empty()
      elif not self.qqueue.empty():
        self.active = True
        while self.active:
          trans = await self.qqueue.get()
          self.bus.we.setimmediatevalue(0)
          self.bus.cs.setimmediatevalue(1)
          self.bus.addr.setimmediatevalue(trans.address)
          trans.data = self.bus.data.value
          await self.rqueue.put(trans)
          self._idle_read.set()
          await RisingEdge(self.clock)

          self.active = not self.qqueue.empty()
      else:
        self.active = False
        self.bus.we.setimmediatevalue(0)
        self.bus.cs.setimmediatevalue(0)
        await RisingEdge(self.clock)
\end{lstlisting}

\section{Architecture}

Please see \ref{Code Documentation} for more information.

\par
busbase is the base class for busbase methods.
\par
noSignal used if a signal does not exist, you can switch it out with this class and still use the value attribute.
\par
transaction is the base class for all transactions. This will contain items such as address/data for read/write.

\subsection{Directory Guide}

\par
Below highlights important folders from the root of the directory.

\begin{enumerate}
  \item \textbf{docs} Contains all documentation related to this project.
    \begin{itemize}
      \item \textbf{manual} Contains user manual and github page that are generated from the latex sources.
    \end{itemize}
  \item \textbf{cocotbext} Contains source files for the extension
    \begin{itemize}
      \item \textbf{busbase} Contains source files for busbase
    \end{itemize}
  \item \textbf{tests} Contains test files for cocotb
\end{enumerate}

\newpage

\section{Simulation}
\par
A simulation for testing the cores can be run to verify operation.

\subsection{cocotb}
\par
To use the cocotb tests you must install the following python libraries.
\begin{lstlisting}[language=bash]
  $ pip install cocotb
  $ pip install -e .
\end{lstlisting}

Then you must enter the tests folder and enter the folder of the type you wish to test. From there you may execute the following command
which will kick off the test.
\begin{lstlisting}[language=bash]
  $ make
\end{lstlisting}

\newpage

\section{Code Documentation} \label{Code Documentation}

\par
Natural docs is used to generate documentation for this project. The next lists the following sections.

\begin{itemize}
  \item \textbf{init} Python init code.\\
  \item \textbf{busbase} Contains bus base for threads and read/write methods.\\
  \item \textbf{cocotb test} Python TestFactory code.\\
  \item \textbf{cocotb verilog test wrapper} Verilog wrapper module.\\
\end{itemize}

